%%%%%%%%%%%%%%%%%%%%%%%%%%%%%%
% LATEX-TEMPLATE PRESENTATIE
%-------------------------------------------------------------------------------
% Voor informatie over presenterenn, zie
% http://practicumav.nl/presenteren/presenteren.html
% Voor readme en meest recente versie van het template, zie
% https://gitlab-fnwi.uva.nl/informatica/LaTeX-template.git
% Gebaseerd op een template van: http://www.LaTeXTemplates.com
% Licentie: CC BY-NC-SA 3.0
% (http://creativecommons.org/licenses/by-nc-sa/3.0/)
%%%%%%%%%%%%%%%%%%%%%%%%%%%%%%

%-------------------------------------------------------------------------------
%	PACKAGES EN CONFIGURATIE
%-------------------------------------------------------------------------------

\documentclass[aspectratio=43]{uva-inf-presentation}
\usepackage[dutch]{babel}

\title{Hoe Bak je Lekkere Frietjes}
\course{Inleiding Frituren}
\assignment{Naam van het samen te vatten boek}
\assignmenttype{Samenvatting}
\authors{Auteur 1; Auteur 2}
\uvanetids{UvAnetID student 1; UvAnetID student 2}
\tutor{Naam van de tutor}
\mentor{}
\docent{}
\group{Naam van de groep}

\begin{document}

\begin{titelframe}
% De eerste slide is de titelpagina
\titlepage
\end{titelframe}

\begin{frame}
% Geeft een inhoudsopgave voor de presentatie
\frametitle{Inhoudsopgave}
\tableofcontents
\end{frame}

%-------------------------------------------------------------------------------
%	PRESENTATIE SLIDES
%-------------------------------------------------------------------------------

% Secties zijn er om structuur in je presentatie aan te brengen
% en worden direct in de inhoudsopgave opgenomen
\section{Eerste sectie}
% Maak een subsection voor een serie slides met een gedeeld thema
\subsection{Voorbeeld subsectie}

\begin{frame}
\frametitle{Tekst}
Sed iaculis dapibus gravida. Morbi sed tortor erat, nec interdum arcu. Sed id
lorem lectus. Quisque viverra augue id sem ornare non aliquam nibh tristique.
Aenean in ligula nisl. Nulla sed tellus ipsum. Donec vestibulum ligula non lorem
vulputate fermentum accumsan neque mollis.\\~\\

Sed diam enim, sagittis nec condimentum sit amet, ullamcorper sit amet libero.
Aliquam vel dui orci, a porta odio. Nullam id suscipit ipsum. Aenean lobortis
commodo sem, ut commodo leo gravida vitae. Pellentesque vehicula ante iaculis
arcu pretium rutrum eget sit amet purus. Integer ornare nulla quis neque
ultrices lobortis. Vestibulum ultrices tincidunt libero, quis commodo erat
ullamcorper id. \end{frame}

%------------------------------------------------

\begin{frame}
\frametitle{Bullet Points}
\begin{itemize}
\item Lorem ipsum dolor sit amet, consectetur adipiscing elit
\item Aliquam blandit faucibus nisi, sit amet dapibus enim tempus eu
\item Nulla commodo, erat quis gravida posuere, elit lacus lobortis est, quis
      porttitor odio mauris at libero
\item Nam cursus est eget velit posuere pellentesque
\item Vestibulum faucibus velit a augue condimentum quis convallis nulla gravida
\end{itemize}
\end{frame}

%------------------------------------------------

\begin{frame}
\frametitle{Blokken Tekst}
\begin{block}{Blok 1}
Lorem ipsum dolor sit amet, consectetur adipiscing elit. Integer lectus nisl,
ultricies in feugiat rutrum, porttitor sit amet augue. Aliquam ut tortor mauris.
Sed volutpat ante purus, quis accumsan dolor.
\end{block}

\begin{block}{Blok 2}
Pellentesque sed tellus purus. Class aptent taciti
sociosqu ad litora torquent per conubia nostra, per inceptos himenaeos.
Vestibulum quis magna at risus dictum tempor eu vitae velit.
\end{block}

\begin{block}{Blok 3}
Suspendisse tincidunt sagittis gravida. Curabitur condimentum, enim sed
venenatis rutrum, ipsum neque consectetur orci, sed blandit justo nisi ac lacus.
\end{block}
\end{frame}

%------------------------------------------------

\begin{frame}
\frametitle{Meerdere Kolommen}
% De "c" zorgt voor gecentreerde verticale uitlijning
\begin{columns}[c]
 verticaal

% Breedte van de linkerkolom
\column{.45\textwidth}
\textbf{Koptekst}
\begin{enumerate}
\item Stelling
\item Uitleg
\item Voorbeeld
\end{enumerate}

% Breedte van de rechterkolom
\column{.5\textwidth}
Lorem ipsum dolor sit amet, consectetur adipiscing elit. Integer lectus nisl,
ultricies in feugiat rutrum, porttitor sit amet augue. Aliquam ut tortor mauris.
Sed volutpat ante purus, quis accumsan dolor.
\end{columns}
\end{frame}

%------------------------------------------------

\section{Tweede sectie}

%------------------------------------------------

\begin{frame}
\frametitle{Tabel}
\begin{table}
\begin{tabular}{l l l}
\toprule
\textbf{Behandeling} & \textbf{Reactie 1} & \textbf{Reactie 2}\\
\midrule
Behandeling 1 & 0.0003262 & 0.562 \\
Behandeling 2 & 0.0015681 & 0.910 \\
Behandeling 3 & 0.0009271 & 0.296 \\
\bottomrule
\end{tabular}
\caption{Tabel onderschrift}
\end{table}
\end{frame}

%------------------------------------------------

\begin{frame}
\frametitle{Stelling}
\begin{theorem}[Massa-energierelatie]
$E = mc^2$
\end{theorem}
\end{frame}

%------------------------------------------------

% Gebruik fragile wanneer verbatim wordt gebruikt
\begin{frame}[fragile]
\frametitle{Verbatim}
\begin{example}[Code van de Stelling slide]
\begin{verbatim}
\begin{frame}
\frametitle{Stelling}
\begin{theorem}[Massa-energierelatie]
$E = mc^2$
\end{theorem}
\end{frame}\end{verbatim}
\end{example}
\end{frame}

%------------------------------------------------

\begin{frame}
\frametitle{Figuur}
Dit is een voorbeeld van een afbeelding
\begin{figure}
\includegraphics[width=0.8\linewidth]{logoUvA_nl}
\end{figure}
\end{frame}

%------------------------------------------------

\begin{frame}
\frametitle{Conclusie}
\Large{\centerline{Belangrijkste conclusie}}
\end{frame}
\end{document}
